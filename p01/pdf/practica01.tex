\documentclass[12pt]{article}

\usepackage[utf8]{inputenc}
\usepackage[shortlabels]{enumitem}
\usepackage[spanish]{babel}
\usepackage[left=3cm,right=2cm,top=2cm,bottom=2cm]{geometry}

\title{Sistema de procesamiento de lenguaje}
\date{\today}
\author{Rafael de Jesús García García}

\begin{document}
  \pagenumbering{arabic}
  \maketitle

  \section{Desarrollo}

    \begin{itemize}
      \item [2.]
      \begin{enumerate}[a)]
        \item El archivo 'stdio.h' se encuentra en /usr/include/stdio.h. Ahí
        se encuentran todas las definiciones de funciones, macros, variables y
        constantes. \\
        Las funciones se pueden clasificar en dos categorías: funciones de
        manipulación de ficheros y funciones de manipulación de entradas y salidas.
        \item El archivo .i contiene todas las definiciones que se encuentran en
        programa.c, la diferencia es que los archivos .h se incluyen (en este caso
        solo tenemos a stdio.h), las macros son reemplazadas y los comentarios
        son eliminados.
        \item Es el programa fuente modificado, lo que está entre el Preprocesador
        y el Compilador.
      \end{enumerate}
      \item [3.]
      \begin{enumerate}[a)]
        \item Activa todos los mensajes de advertencias del compilador.
        \item Generar código ensamblador.
        \item Contiene el programa en ensamblador y la extensión es '.s'.
        \item Al programa objeto en lenguaje ensamblador.
      \end{enumerate}
      \item [4.]
      \begin{enumerate}[a)]
        \item Se llama al PO en lenguaje ensamblador con el ensamblador de GNU.
        \item Código máquina relocalizable.
        \item Código objeto porque se usó la bandera -o
      \end{enumerate}
      \item [5.]
      \begin{itemize}
        \item \detokenize{crt1.o: /usr/lib64/crt1.o}
        \item \detokenize{crti.o: /usr/lib64/crti.o}
        \item \detokenize{crtbegin.o: /usr/lib/gcc/x86_64-redhat-linux/4.8.3/crtbegin.o}
        \item \detokenize{crtend.o: /usr/lib/gcc/x86_64-redhat-linux/4.8.3/crtend.o}
        \item \detokenize{crtn.o: /usr/lib64/crtn.o}
      \end{itemize}
      \item [6.] Debería generarse código máquina, sin embargo el comando no es
      correcto. 
    \end{itemize}
\end{document}
